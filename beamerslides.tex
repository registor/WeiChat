\RequirePackage[l2tabu, orthodox]{nag}
% 包含beamer宏包
\documentclass[t, fragile, xcolor=svgnames]{ctexbeamer}
% 草稿模式,加快编译速度\documentclass[draft,xcolor=svgnames]{beamer}

% 用TiKZ绘制的QQ聊天界面宏包
\usepackage{qqchat}

% 聊天模拟中表情符号命令,注意路径设置
\newcommand{\emoji}[1]{\includegraphics[width=1em]{twitter/#1}}

% 设置标题==================================================
\title[QQ聊天界面] % (可选,仅当标题过长时使用)
{TiKZ绘制的仿QQ聊天界面的模板}
\author[Nine, G.] % (可选,仅当有多个作者时使用)
{
  耿楠
}

\institute[
CS of CIE, NWSUAF\\
Yangling, China ] % 可选项,在每页边栏的底部显示
{% 显示在标题页
  计算机科学系 \\
  信息工程学院 \\
  西北农林科技大学\\
  中国$\boldsymbol{\cdot}$杨凌  
  
}

\date{\today}

% =======================================================

\begin{document}

%%%%%%%%%%%%%%%%%%%%%%%%%%%%%%%%%%%%%%%%%%%%%%%%%%%%%%%%%%%%%%%%% 
% 标题页
\begin{frame}[plain,noframenumbering] % plain选项移除标题页的边栏和页眉
  \titlepage
\end{frame}

\begin{frame}[fragile, t]{C-Story}{趣谈}%, label=testframe
  \begin{itemize}
  \item C语言趣谈
  \end{itemize}
  \vspace{-4ex}
  \begin{center}
    \setPartnerName{说C解C5000人群}% 群名称
    \groupfriend{\QqGroup[1em]{Qblue}{white}}% 群标志
    \setStatus{3668人在线}% 在线状态
    \scalebox{0.45}{
      \begin{qqminipage}        
        \SysTime{15:14}% 系统时间       
        \myMessage{C语言高手横空出世}\\% 换行是必须的,否则排版后续不正确,需要改进
        \time{15:10}% 聊天时间
        \InputMessage{\emoji{1F602}……}% 不可放在后面,位置会出错,与article位置无法一致,需要改进
        \setPartnerPic{imgs/xiaojiejie.jpg}% 更换头像
        \you{秋}% 昵称
        {如何快速学会编程\emoji{1F605}}% 聊天内容
        \setMePic{imgs/hlz}% 更换头像
        \meb{夏}% 昵称
        {求生意志\emoji{1F606}}{imgs/eye}% 聊天内容
        \setPartnerPic{imgs/xuemei}% 更换头像
        \youa{春}% 昵称
        {编程有有技巧吗\emoji{1F64A}}{imgs/2615}% 聊天内容
        \meb{夏}% 昵称
        {有,熟能生巧\emoji{1F600}}{imgs/eye}% 聊天内容
        \setPartnerPic{imgs/xiaogege}% 更换头像
        \youb{冬}% 昵称
        {一编程就打瞌睡,怎么破\emoji{1F607}}{imgs/2615}% 聊天内容
        \meb{夏}% 昵称
        {目测对编程过敏\emoji{1F604}}{imgs/eye}% 聊天内容
        \setPartnerPic{imgs/xuemei}% 更换头像
        \youa{春}% 昵称
        {高手每天都写多少代码啊\emoji{1F628}}{imgs/2615}% 聊天内容
        \meb{夏}% 昵称
        {一两行到 N 行不等,直到吐\emoji{1F624}}{imgs/eye}% 聊天内容
      \end{qqminipage}\qquad
      \begin{qqminipage}        
        \SysTime{15:35}% 系统时间
        \myMessage{C语言高手横空出世}\\% 换行是必须的,否则后续排版不正确,需要改进
        \time{15:25}% 聊天时间
        \InputMessage{\emoji{1F634}……}% 不可放在后面,位置会出错,需要改进
        \setMePic{imgs/hlz}% 更换头像        
        \setPartnerPic{imgs/xiaojiejie.jpg}% 更换头像
        \you{秋}% 昵称
        {您认为的编程高手是什么状态\emoji{1F60B}}% 聊天内容
        \meb{夏}% 昵称
        {目空一切\emoji{1F609}}{imgs/eye}% 聊天内容
        \setPartnerPic{imgs/xiaogege}% 更换头像
        \youb{冬}% 昵称
        {我写的代码老是出错是咋回事\emoji{1F63E}}{imgs/2615}% 聊天内容
        \meb{夏}% 昵称
        {对计算机太凶,温柔点就好了\emoji{1F627}}{imgs/eye}% 聊天内容
        \setPartnerPic{imgs/shimei.png}% 更换头像
        \you{月}% 昵称
        {师,能不能讲一下你的经历\emoji{1F633}}% 聊天内容
        \meb{夏}% 昵称
        {苦、苦、苦,哭、哭、哭,成\emoji{1F637}}{imgs/eye}% 聊天内容
        \you{月}% 昵称
        {想问问,您编程之余学什么\emoji{1F608}}% 聊天内容
        \meb{夏}% 昵称
        {学睡觉\emoji{1F634}}{imgs/eye}% 聊天内容
      \end{qqminipage}
    }
  \end{center}  
\end{frame}

%%%%%%%%%%%%%%%%
\end{document}

%%% Local Variables:
%%% mode: latex
%%% TeX-master: t
%%% End:
